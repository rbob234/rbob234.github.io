\documentclass[paper=a4,fontsize=paper,12.5pt]{book}

\usepackage[margin = 1in,includefoot]{geometry} 
\usepackage{amsmath}
\usepackage{amssymb}
\usepackage{stix}
\usepackage{amsthm}
\usepackage{graphicx}
 \usepackage{tikz}
 \usepackage{polynom}
 \usepackage{hyperref}
 \usepackage{apacite}
 

  
 
 
\usepackage{bibentry}
\usepackage[square,sort,comma,numbers]{natbib}
\usepackage{mathtools}

  \usepackage{natbib}
    \usepackage{chapterbib}
    \sectionbib{\section}{section}

   \DeclareUnicodeCharacter{00A0}{ }
   \makeatletter
\newcommand{\mtmathitem}{%
\xpatchcmd{\item}{\@inmatherr\item}{\relax\ifmmode$\fi}{}{\errmessage{Patching of \noexpand\item failed}}
\xapptocmd{\@item}{$}{}{\errmessage{appending to \noexpand\@item failed}}}
\makeatother

\newenvironment{mathitem}[1][]{%
\itemize[#1]\mtmathitem}{$\endlist}                    %$

\newenvironment{mathenum}[1][]{%
\enumerate[#1]\mtmathitem}{$\endlist}                  %$

\newenvironment{mathdesc}[1][]{%
\description[#1]\mtmathitem}{$\endlist}    

   




\newcommand{\3}{\vspace*{3mm}}
\newcommand{\Proof}{\textit{Proof:}}
\newcommand{\IFF}{$\Longleftrightarrow$ \hspace*{.5mm}}
\newcommand{\RIGHT}{\Longrightarrow \hspace*{.5mm}}
\newcommand{\LEFT}{\Longleftarrow \hspace*{.5mm}}
\newcommand{\coker}{\textnormal{coker} \hspace*{.5mm}}
\newcommand{\im }{\textnormal{im} \hspace*{.5mm}}
\newcommand{\Spec}{\textnormal{Spec}}
\newcommand{\cl}{\textnormal{cl}}
\newcommand{\Solution}{\textnormal{Solution}}
\newcommand{\tn}[1]{\textnormal{#1}}
\newcommand{\Fund}[2]{{\pi}_{1}(#1,#2)}
\newcommand{\fund}[1]{{\pi}_{1}(#1)}
\newcommand{\A}{\fund{\C{1}}}

\newcommand*{\comb}[2]{\binom{#1}{#2}}
\newcommand{\N}{\mathbb{N}}
\newcommand{\Z}{\mathbb{Z}}
\newcommand{\R}{\mathbb{R}}
\newcommand{\C}[1]{{S}^{#1}}
\newcommand{\FIG}[2]{
 \begin{figure}[!hbt]
 \begin{center}
 \begin{minipage}{0.85\textwidth}
 \centering{\includegraphics[width=90mm]{#1}}
 \caption{\label{#1}\small{#2}}
 \end{minipage}
 \end{center}
 \end{figure}
 }


\bibliographystyle{apacite}




\begin{document}




\newtheorem{lemma}{Lemma}
\newtheorem{sublemma}{Lemma}[lemma]
\newtheorem{theorem}{Theorem}
\newtheorem{conjecture}{Conjecture}

\newtheorem{definition}{Definition}[section]
\newtheorem{problem}{Problem}
\newtheorem{corollary}{Corollary}

\section*{Basic Results} 


\begin{definition}

The arithmetic derivative, $D$, is defined recursively in the following manner for all natural numbers $n \in \N$

\3

\begin{enumerate}

\item $D(1) = D(0) = 0$

\item $D(p) = 1$ for any prime $p$

\item For every pair of natural numbers $n$ \&\ $m$, $D(mn) = mD(n) + nD(m)$ 

\end{enumerate}

\end{definition}

\3

The arithmetic derivative is much like the normal derivative operator if one sees the numbers $0$ \&\ $1$ as the "constant" functions, prime numbers as "lines" with slope $1$ and if one required the arithmetic derivative to satisfy the Lebiniz rule. 

\3

Since this definition doesn't allow one to directly compute the derivative of a number quickly, the following lemma shall help us greatly.

\begin{lemma}

If \[x = \prod_{i =1 }^n m_i\] then \[D(x) = x\sum_{i = 1}^n \frac{D(m_i)}{m_i}\]


\end{lemma}

\Proof

\3

We shall proceed by induction on the amount of elements in the product expansion of $x$. If $n = 1$ then $x = m_1$ and thus  \[x\sum_{i = 1}^n \frac{D(m_i)}{m_i} = (m_1)\frac{D(m_1)}{m_1} = D(m_1) = D(x)\]

\3

Now let's assume that for any $x$ with $n$ factors in its product expansion the formula \[D(x) = x\sum_{i = 1}^n \frac{D(m_i)}{m_i}\]  applies. Let $x'$ be equal to $xm_{n+1}$ and thus $D(x') = D(x)m_{n+1} +xD(m_{n+1})$. We can rewrite $D(x)m_{n+1} +xD(m_{n+1})$ as \[xm_{n+1} \sum_{i = 1}^n \frac{D(m_i)}{m_i}+ \frac{x'D(m_{n+1})}{m_{n+1}} = x' \sum_{i = 1}^n \frac{D(m_i)}{m_i} +x'\frac{D(m_{n+1})}{m_{n+1}} \].

\3

We can then factor out the $x'$ and our final expression for $D(x')$ is $x'\sum_{i = 1}^{n+1} \frac{D(m_i)}{m_i}$ which is exactly what we wanted to show $\QED$

\3

With lemma $1$ in hand we can easily write down an expression for $D(x)$ in terms of its prime factors.\[\text{Let} \hspace*{1mm} x = \prod_{i =1}^n{p_i}^{k_i}\] be $x$'s prime factorization.

\[ D(x) = x\sum_{i = 1}^{n} \frac{D({p_i}^{k_i})}{{p_i}^{k_i}} \]

 \3
 
Obviously, $D(p^k)$ is $kp^{k-1}$, since due to lemma $1$ we have \[D(p^k) = p^k\sum_{i = 1}^n \frac{D(p)}{p} =p^k\sum_{i = 1}^n \frac{1}{p} \] and thus \[D(p^k)  = \sum_{i = 1}^n p^{k-1} = k p^{k-1} \]

\3

Finally, we have a fully simplified expression for the arithmetic derivative of $x$.

\[D(x) = x\sum_{i = 1}^{n} \frac{D({p_i}^{k_i})}{{p_i}^{k_i}} = x\sum_{i = 1}^{n} \frac{{k_i}{p_i}^{k_i -1}}{{p_i}^{k_i}} = x\sum_{i = 1}^{n} \frac{k_i}{p_i}\] 

\3

By using the notation $v_p(x)$ to denote the exponent of a prime $p$ in the prime factorization of $x$ the above formula can be rewritten as the following.

\[D(x) = x\sum_{p \mid x} \frac{v_p(x)}{p}\]

\section*{Cool Results}

\subsection*{Differential Equations} 

\3

Let $x'$ be shorthand for $D(x)$ and of course $x''$ will be $D(D(x))$ and so on.

\3

In calculus, the most basic differential equation is $x' = 0$ and the solutions of such an equation are the constant functions, so we would expect the same to be true in this context, and it indeed is.

\begin{theorem}

If $x' = 0$ then $x \in \{0,1\}$


\end{theorem}

\Proof 

By our definition of $D(x)$, $0'$ and $1'$ are both $0$. Now let $x>1$ and $x'$ is equal to $x\sum_{p \mid x} \frac{v_p(x)}{p}$. The only way it can be $0$ is if $\sum_{p \mid x} \frac{v_p(x)}{p}$ is $0$ and that's only true if and only $v_p(x)$ is $0$ for all $p \mid x$ which is obviously false and thus $x'$ is never $0$ if $x>1$ $\QED$

\3

Another simple differential equation is $x' = 1$ which characterizes all linear functions with slope $1$. Similarly, in this context we should see that if $x'$ is $1$ then $x$ is a prime number.

\begin{theorem}

If $x' = 1$ then $x$ is prime 


\end{theorem}

\Proof

























\end{document}