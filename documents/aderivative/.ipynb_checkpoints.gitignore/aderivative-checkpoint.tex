\documentclass[paper=a4,fontsize=paper,12.5pt]{book}

\usepackage[margin = 1in,includefoot]{geometry} 
\usepackage{amsmath}
\usepackage{amssymb}
\usepackage{stix}
\usepackage{amsthm}
\usepackage{graphicx}
 \usepackage{tikz}
 \usepackage{polynom}
 \usepackage{hyperref}
 \usepackage{apacite}
 

  
 
 
\usepackage{bibentry}
\usepackage[square,sort,comma,numbers]{natbib}
\usepackage{mathtools}

  \usepackage{natbib}
    \usepackage{chapterbib}
    \sectionbib{\section}{section}

   \DeclareUnicodeCharacter{00A0}{ }
   \makeatletter
\newcommand{\mtmathitem}{%
\xpatchcmd{\item}{\@inmatherr\item}{\relax\ifmmode$\fi}{}{\errmessage{Patching of \noexpand\item failed}}
\xapptocmd{\@item}{$}{}{\errmessage{appending to \noexpand\@item failed}}}
\makeatother

\newenvironment{mathitem}[1][]{%
\itemize[#1]\mtmathitem}{$\endlist}                    %$

\newenvironment{mathenum}[1][]{%
\enumerate[#1]\mtmathitem}{$\endlist}                  %$

\newenvironment{mathdesc}[1][]{%
\description[#1]\mtmathitem}{$\endlist}    

   




\newcommand{\3}{\vspace*{3mm}}
\newcommand{\Proof}{\textit{Proof:}}
\newcommand{\IFF}{$\Longleftrightarrow$ \hspace*{.5mm}}
\newcommand{\RIGHT}{\Longrightarrow \hspace*{.5mm}}
\newcommand{\LEFT}{\Longleftarrow \hspace*{.5mm}}
\newcommand{\coker}{\textnormal{coker} \hspace*{.5mm}}
\newcommand{\im }{\textnormal{im} \hspace*{.5mm}}
\newcommand{\Spec}{\textnormal{Spec}}
\newcommand{\cl}{\textnormal{cl}}
\newcommand{\Solution}{\textnormal{Solution}}
\newcommand{\tn}[1]{\textnormal{#1}}
\newcommand{\Fund}[2]{{\pi}_{1}(#1,#2)}
\newcommand{\fund}[1]{{\pi}_{1}(#1)}
\newcommand{\A}{\fund{\C{1}}}
\newcommand{\D}{\vert \vert}
\newcommand{\Der}[1]{#1\sum_{p \mid #1} \frac{v_p(#1)}{p}}
\newcommand{\Id}[1]{\sum_{p \mid #1} \frac{v_p(#1)}{p}}
\newcommand{\Par}[2]{\frac{v_#2(#1)}{#2}}


\newcommand*{\comb}[2]{\binom{#1}{#2}}
\newcommand{\N}{\mathbb{N}}
\newcommand{\Z}{\mathbb{Z}}
\newcommand{\R}{\mathbb{R}}
\newcommand{\C}[1]{{S}^{#1}}
\newcommand{\FIG}[2]{
 \begin{figure}[!hbt]
 \begin{center}
 \begin{minipage}{0.85\textwidth}
 \centering{\includegraphics[width=90mm]{#1}}
 \caption{\label{#1}\small{#2}}
 \end{minipage}
 \end{center}
 \end{figure}
 }


\bibliographystyle{apacite}




\begin{document}




\newtheorem{lemma}{Lemma}
\newtheorem{sublemma}{Lemma}[lemma]
\newtheorem{theorem}{Theorem}
\newtheorem{conjecture}{Conjecture}

\newtheorem{definition}{Definition}[section]
\newtheorem{problem}{Problem}
\newtheorem{corollary}{Corollary}

\section*{Basic Results} 


\begin{definition}

The arithmetic derivative, $D$, is defined recursively in the following manner for all natural numbers $n \in \N$

\3

\begin{enumerate}

\item $D(1) = D(0) = 0$

\item $D(p) = 1$ for any prime $p$

\item For every pair of natural numbers $n$ \&\ $m$, $D(mn) = mD(n) + nD(m)$ 

\end{enumerate}

\end{definition}

\3

The arithmetic derivative is much like the normal derivative operator if one sees the numbers $0$ \&\ $1$ as the "constant" functions, prime numbers as "lines" with slope $1$ and if one required the arithmetic derivative to satisfy the Lebiniz rule. 

\3

Since this definition doesn't allow one to directly compute the derivative of a number quickly, the following lemma shall help us greatly.

\begin{lemma}

If \[x = \prod_{i =1 }^n m_i\] then \[D(x) = x\sum_{i = 1}^n \frac{D(m_i)}{m_i}\]


\end{lemma}

\Proof

\3

We shall proceed by induction on the amount of elements in the product expansion of $x$. If $n = 1$ then $x = m_1$ and thus  \[x\sum_{i = 1}^n \frac{D(m_i)}{m_i} = (m_1)\frac{D(m_1)}{m_1} = D(m_1) = D(x)\]

\3

Now let's assume that for any $x$ with $n$ factors in its product expansion the formula \[D(x) = x\sum_{i = 1}^n \frac{D(m_i)}{m_i}\]  applies. Let $x'$ be equal to $xm_{n+1}$ and thus $D(x') = D(x)m_{n+1} +xD(m_{n+1})$. We can rewrite $D(x)m_{n+1} +xD(m_{n+1})$ as \[xm_{n+1} \sum_{i = 1}^n \frac{D(m_i)}{m_i}+ \frac{x'D(m_{n+1})}{m_{n+1}} = x' \sum_{i = 1}^n \frac{D(m_i)}{m_i} +x'\frac{D(m_{n+1})}{m_{n+1}} \].

\3

We can then factor out the $x'$ and our final expression for $D(x')$ is $x'\sum_{i = 1}^{n+1} \frac{D(m_i)}{m_i}$ which is exactly what we wanted to show $\QED$

\3

With lemma $1$ in hand we can easily write down an expression for $D(x)$ in terms of its prime factors.\[\text{Let} \hspace*{1mm} x = \prod_{i =1}^n{p_i}^{k_i}\] be $x$'s prime factorization.

\[ D(x) = x\sum_{i = 1}^{n} \frac{D({p_i}^{k_i})}{{p_i}^{k_i}} \]

 \3
 
Obviously, $D(p^k)$ is $kp^{k-1}$, since due to lemma $1$ we have \[D(p^k) = p^k\sum_{i = 1}^n \frac{D(p)}{p} =p^k\sum_{i = 1}^n \frac{1}{p} \] and thus \[D(p^k)  = \sum_{i = 1}^n p^{k-1} = k p^{k-1} \]

\3

Finally, we have a fully simplified expression for the arithmetic derivative of $x$.

\[D(x) = x\sum_{i = 1}^{n} \frac{D({p_i}^{k_i})}{{p_i}^{k_i}} = x\sum_{i = 1}^{n} \frac{{k_i}{p_i}^{k_i -1}}{{p_i}^{k_i}} = x\sum_{i = 1}^{n} \frac{k_i}{p_i}\] 

\3

By using the notation $v_p(x)$ to denote the exponent of a prime $p$ in the prime factorization of $x$ the above formula can be rewritten as the following.

\begin{theorem}

\[D(x) = x\sum_{p \mid x} \frac{v_p(x)}{p}\]

\end{theorem}

\section*{Cool Results}

\subsection*{Differential Equations} 

\3

Let $x'$ be shorthand for $D(x)$ and of course $x''$ will be $D(D(x))$ and so on.

\3

In calculus, the most basic differential equation is $x' = 0$ and the solutions of such an equation are the constant functions, so we would expect the same to be true in this context, and it indeed is.

\begin{theorem}

If $x' = 0$ then $x \in \{0,1\}$


\end{theorem}

\Proof 

By our definition of $D(x)$, $0'$ and $1'$ are both $0$. Now let $x>1$ and $x'$ is equal to $x\sum_{p \mid x} \frac{v_p(x)}{p}$. The only way it can be $0$ is if $\sum_{p \mid x} \frac{v_p(x)}{p}$ is $0$ and that's only true if and only $v_p(x)$ is $0$ for all $p \mid x$ which is obviously false and thus $x'$ is never $0$ if $x>1$ $\QED$

\3

Another simple differential equation is $x' = 1$ which characterizes all linear functions with slope $1$. Similarly, in this context we should see that if $x'$ is $1$ then $x$ is a prime number. Before I proceed with the proof, I will present two useful corollaries, whose proofs are obvious, and one lemma. 

\3

\begin{corollary}
 
  $x>1$ \IFF $x' \geq 1$


\end{corollary}

\3


\begin{corollary}

If $x = p^km$ then $x' = p^{k-1}(km + pm')$ 


\end{corollary}

\3

\begin{lemma}

If $p^k \D x $ and $1\leq k \leq p-1$ then $p^{k-1} \D x'$

\end{lemma}

\3

\begin{theorem}

If $x' = 1$ then $x$ is prime 


\end{theorem}

\Proof
 
By corollary $1$ we have that $x>1$ and let's say that $x$ is not prime. Thus we can rewrite $x$ as $p^k m$ where $p \nmid m$ and $k>0$. Now let's assume that $k\geq p$ then by theorem $1$ we have that $x' = \Der{x}$ and the value of $\Id{x}$ is larger than $1$, since we know that there is at least one $\Par{x}{p}$ term which is larger than or equal to $1$ and thus $x' \geq x$ and since $x>1$ and along with the fact that $x' =1$ we have a contradiction and thus $k \leq p-1$. Now, since we assumed that $0<k$ and we can apply corollary $2$ and we can see quite easily that $p^{k-1} \D x' = 1$ and thus $k-1$ = 0 and thus $k=1$. So $x$ is of the form $pm$ where $p$ is a prime that doesn't divide $m$. Finally, let's assume that $m>1$ and now we symbolically differentiate $x$, getting $pm' + m$, and since $m\geq 2$ \vspace*{.05mm} (and thus $m' \geq1$) and $p \geq2$ we have that $x =  pm' + m \geq 2(1) + 2 = 4 $ and thus $m =1$ and we have that $x =p$ where $p$ is some prime $\QED$
 

\3

The final result of this section will solve the next classical differential equation which is $x' = x$

\begin{theorem}

If $x' =x $ then $x = p^p$ where $p$ is prime or $x$ is $0$


\end{theorem}

\Proof

Let's assume that $p \mid x$ and let $x = p^km$ so we have that $k\geq 1$ and let's also assume that $m\geq 2$. We also choose $p$ such that $k \geq p$ since by corollary $2$,  if $k <p $ we have that $p^{k-1} \D x' = x$ and thus $k$ must be greater than or equal to $p$. By corollary $2$, we have that $x' = \frac{x}{pm}(pm' + km) \geq \frac{x}{pm}(p + pm) = \frac{m+1}{m}x$. And since $x' = x$, $m$ must be $1$ so $x$ is of the form $p^k$. Finally, let's assume that $k>p$ and then we have that $x' = kp^{k-1} >  p(p^{k-1}) = p^{k} = x$ and thus $k = p$, so we have that $x = p^p$ where $p$ is prime $\QED$


\subsection*{Expiremental Results} 

Many results regarding differential equations in this context are quite frankly impossible to prove, due to the fact that the arithmetic derivative is intimately connected to the prime factorization of a number and that not much is known about the prime factorization of the sum of two integers, given the prime factorizations of both integers. So, we resort to finding the solutions of most differential equations via brute force. This section includes some interesting results. 

Now, one metric in number theory is called the natural density of a subset of the naturals.

\begin{definition}

Given $A \susbeteq \N$ the natural density of $A$, $d(A)$, is equal to the following limit, given that it exists 

\[ \lim_{n \to \infty} \frac{\vert A \bigcap [1,n] \vert}{n} \]

\end{definition}


















\end{document}